\chapter{Introduction}

The automation of the implementation of the Docker container serves as a core foundation enabling scalability, security, ´reproducibility and rapid innovation. In the Context of the project like Somnolink provides not only an efficient service operation but also provides the foundation for evidence-based development, regulatory compliance, and effective research collaboration.\\
Docker streamlines the development lifecycle by allowing developers to work in the standardized environments using local containers which provide your applications and services. Container are great for continuous delivery workflow.
This work contributes meaningfully to the field of medical informatics, through this approach, it can be used in automated patient data processing, medical AI Workflow, and secure data integration. It also provides a fast and efficient way to create new operations or updates, which can have positively a huge impact of the patient diagnosis or on the patient care in general.\\Automation is the application of technology, programs, robotics or processes to achieve outcomes with minimal human input, and in the case of the implementation of the Docker container it describes the process of the Deployment, the Configuration and the actualisation of the Software use and their dependencies with the help of the Docker Container and REST APIS based technologies with reduced human assistance. 
while a large scale of previous studies has used the methods of the Docker Container the  in the field of neuro-imaging,telemedicine applications and in the electronic health record systems.\footnote{Collabnix, 5 benefits of Docker for the healthcare industry, 09.10.2023, https://collabnix.com/5-benefits-of-docker-for-the-healthcare-industry (Zugriff am 17.07.2025)}.One considerable study in this context we can recite the study of the integration of the artificial intelligence tools in the Clinical Research Setting: The Ovarian Cancer Use Case by Lorena Escudero Sanchez\footnote{Lorena Escudero Sanchez et al., Integrating Artificial Intelligence Tools in the Clinical Research Setting: The Ovarian Cancer Use Case, in: Diagnostics, Vol. 13(17), 2023, Article 2813, https://doi.org/10.3390/diagnostics13172813 (Zugriff am 17.07.2025)}. In their 2023 study they proposed using a integration of the AI tool and one of the methods used were the Docker-Xnat pipeline. the developed tool were very powerful and were able to find tumors and precily. In addition by Satrajit Chakrabarty \footnote{et al., Deep learning‑based end‑to‑end scan‑type classification, pre‑processing, and segmentation of clinical neuro‑oncology studies, in: Proceedings of SPIE, Vol.12469 (2023), Article 124690N, https://doi.org/10.1117/12.2647656 (Zugriff am 17.07.2025)} who implemented a  AI based framework that can automate a MRI scan tumor segmentation and characterization using the  Docker Containers in xnat.\\ 
The Docker Container has a huge rule for the devolopment and integration of virtuell imaging platforms\footnote{Satrajit Chakrabarty et al., Container applications for the development and integration of virtual imaging trials, in: Medical Physics, 2025, https://doi.org/10.1002/mp.17777 (Zugriff am 17.07.2025)} they allow the the abstraction, installation, and configuration of environments so that software can be both distributed in self-contained images and used repeatably by tool consumers.\footnote{SatrajitChakrabarty et al., Container applications for the development and integration of virtual imaging trials, in: Medical Physics, 2025, https://doi.org/10.1002/mp.17777 (Zugriff am17.07.2025)}\\ Despite the flexibility of xnat of integrating external tool (AI, pipelines or plugins) the process of integrating the Docker Containers remains a challenge and requires often steps and manual scripting. Several studies has been using the implementation  of Docker Containerization but few of them have addressed  the idea to automatize the idea of the implementation on xnat, in order to address this gaps limits the current study propose a Docker-Container-based framework integrated in xnat that enable the fully automation of the Docker implementation in xnat. 
\\
 This project is about  to provide a automation for the process of the implementation of the docker container in xnat. therefore in order to achieve this goal a significant workload remains.
Most of the cases if we want to implement a docker container in xnat we proceed by writing a external script that the container will then contain, assuming that we are  working on xnat  with the Container Service plugin already installed. Afterward we start writting a dockerfile.\footnote{https://docs.docker.com/reference/dockerfile/} Before deploying the container in xnat a JSON Command must be written. After this the container can be without any problem in xnat deployed launched. The primary goal of the the automation is to to summarize all of these steps into  a single script which receives an external script from the user for each creation of a container.\\
 


 

